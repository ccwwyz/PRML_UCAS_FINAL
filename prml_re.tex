% 若编译失败,且生成 .synctex(busy) 辅助文件,可能有两个原因:
% 1. 需要插入的图片不存在:Ctrl + F 搜索 'figure' 将这些代码注释/删除掉即可
% 2. 路径/文件名含中文或空格:更改路径/文件名即可

% ------------------------------------------------------------- %
% >> ------------------ 文章宏包及相关设置 ------------------ << %
% 设定文章类型与编码格式
\documentclass[UTF8]{report}		

% 本文特殊宏包
\usepackage{siunitx} % 埃米单位

% 本 .tex 专属的宏定义
    \def\V{\ \mathrm{V}}
    \def\mV{\ \mathrm{mV}}
    \def\kV{\ \mathrm{KV}}
    \def\KV{\ \mathrm{KV}}
    \def\MV{\ \mathrm{MV}}
    \def\A{\ \mathrm{A}}
    \def\mA{\ \mathrm{mA}}
    \def\kA{\ \mathrm{KA}}
    \def\KA{\ \mathrm{KA}}
    \def\MA{\ \mathrm{MA}}
    \def\O{\ \Omega}
    \def\mO{\ \Omega}
    \def\kO{\ \mathrm{K}\Omega}
    \def\KO{\ \mathrm{K}\Omega}
    \def\MO{\ \mathrm{M}\Omega}
    \def\Hz{\ \mathrm{Hz}}

% 自定义宏定义
    \def\N{\mathbb{N}}
    \def\F{\mathbb{F}}
    \def\Z{\mathbb{Z}}
    \def\Q{\mathbb{Q}}
    \def\R{\mathbb{R}}
    \def\C{\mathbb{C}}
    \def\T{\mathbb{T}}
    \def\S{\mathbb{S}}
    \def\A{\mathbb{A}}
    \def\I{\mathscr{I}}
    \def\Im{\mathrm{Im\,}}
    \def\Re{\mathrm{Re\,}}
    \def\d{\mathrm{d}}
    \def\p{\partial}

% 导入基本宏包
    \usepackage[UTF8]{ctex}     % 设置文档为中文语言
    \usepackage[colorlinks, linkcolor=blue, anchorcolor=blue, citecolor=blue, urlcolor=blue]{hyperref}  % 宏包:自动生成超链接 (此宏包与标题中的数学环境冲突)
    % \usepackage{hyperref}  % 宏包:自动生成超链接 (此宏包与标题中的数学环境冲突)
    % \hypersetup{
    %     colorlinks=true,    % false:边框链接 ; true:彩色链接
    %     citecolor={blue},    % 文献引用颜色
    %     linkcolor={blue},   % 目录 (我们在目录处单独设置),公式,图表,脚注等内部链接颜色
    %     urlcolor={orange},    % 网页 URL 链接颜色,包括 \href 中的 text
    %     % cyan 浅蓝色 
    %     % magenta 洋红色
    %     % yellow 黄色
    %     % black 黑色
    %     % white 白色
    %     % red 红色
    %     % green 绿色
    %     % blue 蓝色
    %     % gray 灰色
    %     % darkgray 深灰色
    %     % lightgray 浅灰色
    %     % brown 棕色
    %     % lime 石灰色
    %     % olive 橄榄色
    %     % orange 橙色
    %     % pink 粉红色
    %     % purple 紫色
    %     % teal 蓝绿色
    %     % violet 紫罗兰色
    % }

    % \usepackage{docmute}    % 宏包:子文件导入时自动去除导言区,用于主/子文件的写作方式,\include{./51单片机笔记}即可。注:启用此宏包会导致.tex文件capacity受限。
    \usepackage{amsmath}    % 宏包:数学公式
    \usepackage{mathrsfs}   % 宏包:提供更多数学符号
    \usepackage{amssymb}    % 宏包:提供更多数学符号
    \usepackage{pifont}     % 宏包:提供了特殊符号和字体
    \usepackage{extarrows}  % 宏包:更多箭头符号
    \usepackage{multicol}   % 宏包:支持多栏 
    \usepackage{graphicx}   % 宏包:插入图片
    \usepackage{float}      % 宏包:设置图片浮动位置
    %\usepackage{article}    % 宏包:使文本排版更加优美
    \usepackage{tikz}       % 宏包:绘图工具
    %\usepackage{pgfplots}   % 宏包:绘图工具
    \usepackage{enumerate}  % 宏包:列表环境设置
    \usepackage{enumitem}   % 宏包:列表环境设置
    \usepackage{longtable}  % 宏包:长表格

% 文章页面margin设置
    \usepackage[a4paper]{geometry}
        \geometry{top=1in}
        \geometry{bottom=1in}
        \geometry{left=0.75in}
        \geometry{right=0.75in}   % 设置上下左右页边距
        \geometry{marginparwidth=1.75cm}    % 设置边注距离(注释、标记等)

% 定义 solution 环境
\usepackage{amsthm}
\newtheorem{solution}{Solution}
        \geometry{bottom=1in}
        \geometry{left=0.75in}
        \geometry{right=0.75in}   % 设置上下左右页边距
        \geometry{marginparwidth=1.75cm}    % 设置边注距离(注释、标记等)

% 配置数学环境
    \usepackage{amsthm} % 宏包:数学环境配置
    % theorem-line 环境自定义
        \newtheoremstyle{MyLineTheoremStyle}% <name>
            {11pt}% <space above>
            {11pt}% <space below>
            {}% <body font> 使用默认正文字体
            {}% <indent amount>
            {\bfseries}% <theorem head font> 设置标题项为加粗
            {:}% <punctuation after theorem head>
            {.5em}% <space after theorem head>
            {\textbf{#1}\thmnumber{#2}\ \ (\,\textbf{#3}\,)}% 设置标题内容顺序
        \theoremstyle{MyLineTheoremStyle} % 应用自定义的定理样式
        \newtheorem{LineTheorem}{Theorem.\,}
    % theorem-block 环境自定义
        \newtheoremstyle{MyBlockTheoremStyle}% <name>
            {11pt}% <space above>
            {11pt}% <space below>
            {}% <body font> 使用默认正文字体
            {}% <indent amount>
            {\bfseries}% <theorem head font> 设置标题项为加粗
            {:\\ \indent}% <punctuation after theorem head>
            {.5em}% <space after theorem head>
            {\textbf{#1}\thmnumber{#2}\ \ (\,\textbf{#3}\,)}% 设置标题内容顺序
        \theoremstyle{MyBlockTheoremStyle} % 应用自定义的定理样式
        \newtheorem{BlockTheorem}[LineTheorem]{Theorem.\,} % 使用 LineTheorem 的计数器
    % definition 环境自定义
        \newtheoremstyle{MySubsubsectionStyle}% <name>
            {11pt}% <space above>
            {11pt}% <space below>
            {}% <body font> 使用默认正文字体
            {}% <indent amount>
            {\bfseries}% <theorem head font> 设置标题项为加粗
           % {:\\ \indent}% <punctuation after theorem head>
            {\\\indent}
            {0pt}% <space after theorem head>
            {\textbf{#3}}% 设置标题内容顺序
        \theoremstyle{MySubsubsectionStyle} % 应用自定义的定理样式
        \newtheorem{definition}{}

%宏包:有色文本框(proof环境)及其设置
    \usepackage[dvipsnames,svgnames]{xcolor}    %设置插入的文本框颜色
    \usepackage[strict]{changepage}     % 提供一个 adjustwidth 环境
    \usepackage{framed}     % 实现方框效果
        \definecolor{graybox_color}{rgb}{0.95,0.95,0.96} % 文本框颜色。修改此行中的 rgb 数值即可改变方框纹颜色,具体颜色的rgb数值可以在网站https://colordrop.io/ 中获得。(截止目前的尝试还没有成功过,感觉单位不一样)(找到喜欢的颜色,点击下方的小眼睛,找到rgb值,复制修改即可)
        \newenvironment{graybox}{%
        \def\FrameCommand{%
        \hspace{1pt}%
        {\color{gray}\small \vrule width 2pt}%
        {\color{graybox_color}\vrule width 4pt}%
        \colorbox{graybox_color}%
        }%
        \MakeFramed{\advance\hsize-\width\FrameRestore}%
        \noindent\hspace{-4.55pt}% disable indenting first paragraph
        \begin{adjustwidth}{}{7pt}%
        \vspace{2pt}\vspace{2pt}%
        }
        {%
        \vspace{2pt}\end{adjustwidth}\endMakeFramed%
        }



% 外源代码插入设置
    % matlab 代码插入设置
    \usepackage{matlab-prettifier}
        \lstset{style=Matlab-editor}    % 继承 matlab 代码高亮 , 此行不能删去
    \usepackage[most]{tcolorbox} % 引入tcolorbox包 
    \usepackage{listings} % 引入listings包
        \tcbuselibrary{listings, skins, breakable}
        \newfontfamily\codefont{Consolas} % 定义需要的 codefont 字体
        \lstdefinestyle{MatlabStyle_inc}{   % 插入代码的样式
            language=Matlab,
            basicstyle=\small\ttfamily\codefont,    % ttfamily 确保等宽 
            breakatwhitespace=false,
            breaklines=true,
            captionpos=b,
            keepspaces=true,
            numbers=left,
            numbersep=15pt,
            showspaces=false,
            showstringspaces=false,
            showtabs=false,
            tabsize=2,
            xleftmargin=15pt,   % 左边距
            %frame=single, % single 为包围式单线框
            frame=shadowbox,    % shadowbox 为带阴影包围式单线框效果
            %escapeinside=``,   % 允许在代码块中使用 LaTeX 命令 (此行无用)
            %frameround=tttt,    % tttt 表示四个角都是圆角
            framextopmargin=0pt,    % 边框上边距
            framexbottommargin=0pt, % 边框下边距
            framexleftmargin=5pt,   % 边框左边距
            framexrightmargin=5pt,  % 边框右边距
            rulesepcolor=\color{red!20!green!20!blue!20}, % 阴影框颜色设置
            %backgroundcolor=\color{blue!10}, % 背景颜色
        }
        \lstdefinestyle{MatlabStyle_src}{   % 插入代码的样式
            language=Matlab,
            basicstyle=\small\ttfamily\codefont,    % ttfamily 确保等宽 
            breakatwhitespace=false,
            breaklines=true,
            captionpos=b,
            keepspaces=true,
            numbers=left,
            numbersep=15pt,
            showspaces=false,
            showstringspaces=false,
            showtabs=false,
            tabsize=2,
        }
        \newtcblisting{matlablisting}{
            %arc=2pt,        % 圆角半径
            % 调整代码在 listing 中的位置以和引入文件时的格式相同
            top=0pt,
            bottom=0pt,
            left=-5pt,
            right=-5pt,
            listing only,   % 此句不能删去
            listing style=MatlabStyle_src,
            breakable,
            colback=white,   % 选一个合适的颜色
            colframe=black!0,   % 感叹号后跟不透明度 (为 0 时完全透明)
        }
        \lstset{
            style=MatlabStyle_inc,
        }



% table 支持
    \usepackage{booktabs}   % 宏包:三线表
    %\usepackage{tabularray} % 宏包:表格排版
    %\usepackage{longtable}  % 宏包:长表格
    %\usepackage[longtable]{multirow} % 宏包:multi 行列


% figure 设置
\usepackage{graphicx}   % 支持 jpg, png, eps, pdf 图片 
\usepackage{float}      % 支持 H 选项
\usepackage{svg}        % 支持 svg 图片
\usepackage{subcaption} % 支持子图
\svgsetup{
        % 指向 inkscape.exe 的路径
       inkscapeexe = C:/aa_MySame/inkscape/bin/inkscape.exe, 
        % 一定程度上修复导入后图片文字溢出几何图形的问题
       inkscapelatex = false                 
   }

% 图表进阶设置
    \usepackage{caption}    % 图注、表注
        \captionsetup[figure]{name=图}  
        \captionsetup[table]{name=表}
        \captionsetup{
            labelfont=bf, % 设置标签为粗体
            textfont=bf,  % 设置文本为粗体
            font=small  
        }
    \usepackage{float}     % 图表位置浮动设置 
        % \floatstyle{plaintop} % 设置表格标题在表格上方
        % \restylefloat{table}  % 应用设置


% 圆圈序号自定义
    \newcommand*\circled[1]{\tikz[baseline=(char.base)]{\node[shape=circle,draw,inner sep=0.8pt, line width = 0.03em] (char) {\small \bfseries #1};}}   % TikZ solution


% 列表设置
    \usepackage{enumitem}   % 宏包:列表环境设置
        \setlist[enumerate]{
            label=\bfseries(\arabic*) ,   % 设置序号样式为加粗的 (1) (2) (3)
            ref=\arabic*, % 如果需要引用列表项,这将决定引用格式(这里仍然使用数字)
            itemsep=0pt, parsep=0pt, topsep=0pt, partopsep=0pt, leftmargin=3.5em} 
        \setlist[itemize]{itemsep=0pt, parsep=0pt, topsep=0pt, partopsep=0pt, leftmargin=3.5em}
        \newlist{circledenum}{enumerate}{1} % 创建一个新的枚举环境  
        \setlist[circledenum,1]{  
            label=\protect\circled{\arabic*}, % 使用 \arabic* 来获取当前枚举计数器的值,并用 \circled 包装它  
            ref=\arabic*, % 如果需要引用列表项,这将决定引用格式(这里仍然使用数字)
            itemsep=0pt, parsep=0pt, topsep=0pt, partopsep=0pt, leftmargin=3.5em
        }  

% 文章默认字体设置
    \usepackage{fontspec}   % 宏包:字体设置
        \setmainfont{STKaiti}    % 设置中文字体为宋体字体
        \setCJKmainfont[AutoFakeBold=3]{STKaiti} % 设置加粗字体为 STKaiti 族,AutoFakeBold 可以调整字体粗细
        \setmainfont{Times New Roman} % 设置英文字体为Times New Roman


% 其它设置
    % 脚注设置
    \renewcommand\thefootnote{\ding{\numexpr171+\value{footnote}}}
    % 参考文献引用设置
        \bibliographystyle{unsrt}   % 设置参考文献引用格式为unsrt
        \newcommand{\upcite}[1]{\textsuperscript{\cite{#1}}}     % 自定义上角标式引用
    % 文章序言设置
        \newcommand{\cnabstractname}{序言}
        \newenvironment{cnabstract}{%
            \par\Large
            \noindent\mbox{}\hfill{\bfseries \cnabstractname}\hfill\mbox{}\par
            \vskip 2.5ex
            }{\par\vskip 2.5ex}


% 各级标题自定义设置
    \usepackage{titlesec}   
    % chapter
        \titleformat{\chapter}[hang]{\normalfont\Large\bfseries\centering}{Chapter \thechapter }{10pt}{}
        \titlespacing*{\chapter}{0pt}{-30pt}{10pt} % 控制上方空白的大小
    % section
        \titleformat{\section}[hang]{\normalfont\large\bfseries}{\thesection}{8pt}{}
    % subsection
        %\titleformat{\subsubsection}[hang]{\normalfont\bfseries}{}{8pt}{}
    % subsubsection
        %\titleformat{\subsubsection}[hang]{\normalfont\bfseries}{}{8pt}{}


% >> ------------------ 文章宏包及相关设置 ------------------ << %
% ------------------------------------------------------------- %



% ----------------------------------------------------------- %
% >> --------------------- 文章信息区 --------------------- << %
% 页眉页脚设置

\usepackage{fancyhdr}   %宏包:页眉页脚设置
    \pagestyle{fancy}
    \fancyhf{}
    \cfoot{\thepage}
    \renewcommand\headrulewidth{1pt}
    \renewcommand\footrulewidth{0pt}
    \chead{模式识别与机器学习大作业}
    \lhead{PRML}
    \rhead{山河湖东南通队}

%文档信息设置
\title{模式识别与机器学习大作业\\ PRML}
\author{王湑~尹超~伍昱衡~陈志强~崔祯徐~郑子辰 \\ (按照姓氏笔画排序) \\\footnotesize 中国科学院大学,北京 10004 \\ \footnotesize University of Chinese Academy of Sciences, Beijing 100049, China}
\date{\footnotesize 2025.6.3 - 2025.6.30} % 设置日期
% >> --------------------- 文章信息区 --------------------- << %
% ----------------------------------------------------------- %     


% 开始编辑文章

\begin{document}
\zihao{5}           % 设置全文字号大小

% --------------------------------------------------------------- %
% >> --------------------- 封面序言与目录 --------------------- << %
% 封面
    \maketitle\newpage  
    \pagenumbering{Roman} % 页码为大写罗马数字
    \thispagestyle{fancy}   % 显示页码、页眉等

% 序言
    \begin{cnabstract}\normalsize 
        本文为笔者模式识别与机器学习的大作业。\par
        望老师批评指正。
    \end{cnabstract}
    \addcontentsline{toc}{chapter}{序言} % 手动添加为目录

% % 不换页目录
%     \setcounter{tocdepth}{0}
%     \noindent\rule{\textwidth}{0.1em}   % 分割线
%     \noindent\begin{minipage}{\textwidth}\centering 
%         \vspace{1cm}
%         \tableofcontents\thispagestyle{fancy}   % 显示页码、页眉等   
%     \end{minipage}  
%     \addcontentsline{toc}{chapter}{目录} % 手动添加为目录

% 目录
\setcounter{tocdepth}{4}                % 目录深度(为1时显示到section)
\tableofcontents                        % 目录页
\addcontentsline{toc}{chapter}{目录}    % 手动添加此页为目录
\thispagestyle{fancy}                   % 显示页码、页眉等 

% 收尾工作
    \newpage    
    \pagenumbering{arabic} 

% >> --------------------- 封面序言与目录 --------------------- << %
% --------------------------------------------------------------- %


\chapter{神经网络部分}

\section{神经网络简介}

\subsection{关键要点}
\begin{itemize}
    \item 神经网络是一种模拟人类大脑的计算模型,用于模式识别和预测。
    \item 不同结构如CNN、RNN、Transformer各有专长,适合不同任务。
    \item 研究表明,选择结构取决于数据类型和任务复杂性。
\end{itemize}

\subsection{神经网络简介}
神经网络(Neural Networks)是一种受生物神经系统启发的机器学习模型,广泛用于分类、回归和生成任务。它由多个节点(神经元)组成,这些节点通过加权连接传递信息,通过训练调整权重以学习数据模式。训练过程包括前向传播、损失计算和反向传播。

\subsection{不同结构概述}
神经网络的结构多样化,每种结构针对特定问题设计。以下是主要类型及其适用场景:
\begin{itemize}
    \item \textbf{前向神经网络(FNN)}:适合静态数据,如基本分类。
    \item \textbf{卷积神经网络(CNN)}:专为图像处理设计,擅长提取空间特征。
    \item \textbf{循环神经网络(RNN)和LSTM}:处理序列数据,如语言和时间序列。
    \item \textbf{Transformer}:用于自然语言处理,处理长距离依赖。
    \item \textbf{生成对抗网络(GAN)}:生成新数据,如图像生成。
\end{itemize}

\subsection{结构差异}
不同结构在数据处理方式和复杂性上存在显著差异。例如,CNN通过卷积层提取图像特征,而RNN通过循环捕捉时间依赖。Transformer则依赖注意力机制,适合并行计算。

\subsection{详细调研笔记}

\subsubsection{神经网络的定义与基本原理}
神经网络是一种计算模型,模仿人类大脑神经系统的结构和功能,由多个层组成,包括输入层、隐藏层和输出层。每个神经元通过加权连接接收输入,应用激活函数(如ReLU、Sigmoid)引入非线性,并传递信号。训练过程通过前向传播计算输出,反向传播调整权重以最小化损失函数(如均方误差或交叉熵)。

根据GeeksforGeeks的《Neural Networks: A Beginner's Guide》(\url{https://www.geeksforgeeks.org/neural-networks-a-beginners-guide/}),神经网络的学习过程包括输入计算、输出生成和参数迭代优化,广泛应用于模式识别和复杂问题解决。

\subsubsection{不同神经网络结构的分类与差异}
神经网络的结构多样化,以下是主要类型及其特点,基于V7Labs的《The Essential Guide to Neural Network Architectures》(\url{https://www.v7labs.com/blog/neural-network-architectures-guide/})和Wikipedia的《Neural Network (Machine Learning)》(\url{https://en.wikipedia.org/wiki/Neural_network_(machine_learning)})的综合分析:

\begin{table}[h]
\centering
\caption{神经网络结构对比}
\begin{tabular}{l p{3.5cm} p{4cm} p{3cm} p{3cm}}
\toprule
\textbf{结构} & \textbf{描述} & \textbf{关键特点} & \textbf{局限性} & \textbf{适用场景} \\
\midrule
FNN & 数据单向流动,无循环 & 无反馈机制,适合静态数据 & 无法处理序列数据 & 基本分类、回归 \\
MLP & FNN扩展,含隐藏层 & 处理非线性,学习复杂特征 & 计算量较大 & 图像分类、语音识别 \\
CNN & 使用卷积和池化层 & 参数共享,提取空间特征 & 池化丢失空间关系 & 图像分析、物体检测、NLP \\
RNN & 处理序列,循环连接 & 记忆功能,捕捉时间依赖 & 梯度消失,训练慢 & NLP、时间序列预测 \\
LSTM & RNN增强,记忆单元 & 解决长序列梯度消失 & 训练速度慢 & 语音识别、机器翻译 \\
GAN & 生成器与判别器对抗 & 生成新数据,如图像、文本 & 训练不稳定 & 图像生成、数据增强 \\
Transformer & 基于注意力机制 & 处理长距离依赖,适合并行计算 & 计算复杂度高 & NLP、机器翻译 \\
ResNet & 深层网络,跳跃连接 & 解决梯度消失,深层训练 & 高计算资源 & 图像分类、目标检测 \\
Hopfield网络 & 基于Hebbian学习 & 能量函数驱动,模式检索 & 不适合训练 & 模式识别、记忆任务 \\
Boltzmann机 & 无监督,生成式模型 & 随机能量函数,生成任务 & 训练复杂 & 深度生成模型 \\
RBF网络 & 功能近似,2013年引入 & 最佳近似,非线性识别 & 结构与MLP不同 & 分类、非线性系统 \\
Highway网络 & 2015年,开放门控 & 训练超深网络,解决退化 & 与ResNet类似 & 深层网络训练 \\
Capsule网络 & 改进CNN,保留层次 & 本地胶囊,旋转鲁棒性 & 实现复杂 & 空间关系处理 \\
MobileNet & 轻量级,适合移动设备 & 深度可分离卷积 & 性能受限 & 移动设备、机器人 \\
\bottomrule
\end{tabular}
\end{table}

\subsubsection{结构之间的关键差异}
\begin{itemize}
    \item \textbf{数据类型}:CNN适合空间数据(如图像),RNN/LSTM适合序列数据(如文本、时间序列),Transformer适合长文本,GAN专注于生成数据。
    \item \textbf{处理方式}:FNN和MLP是静态的,RNN/LSTM有记忆,Transformer使用自注意力机制,CNN通过卷积提取特征。
    \item \textbf{复杂性}:FNN简单,ResNet和Transformer更复杂,适合更深的网络和复杂任务。
    \item \textbf{训练难度}:RNN存在梯度消失,LSTM和ResNet通过设计解决此问题,Transformer依赖大规模数据和计算资源。
\end{itemize}

根据MyGreatLearning的《Types of Neural Networks and Definition of Neural Network》(\url{https://www.mygreatlearning.com/blog/types-of-neural-networks/}),不同结构的生物启发设计(如ANN模仿神经元)决定了其在复杂应用中的表现。

\subsubsection{应用场景与局限性}
\begin{itemize}
    \item \textbf{CNN}:如V7Labs的《Convolutional Neural Networks Guide》
          (\href{https://www.v7labs.com/blog/convolutional-neural-networks-guide/}{链接})所示,
          广泛用于图像分类和物体检测,但池化可能丢失空间信息。
    \item \textbf{RNN和LSTM}:如V7Labs的《Recurrent Neural Networks Guide》
          (\href{https://www.v7labs.com/blog/recurrent-neural-networks-guide/}{链接})所述,
          适合NLP和时间序列,但训练慢,LSTM缓解了长序列问题。
    \item \textbf{Transformer}:如《Attention Is All You Need》
          (\href{https://arxiv.org/abs/1706.03762}{链接})所示,
          主导NLP领域,但计算成本高。
    \item \textbf{GAN}:如《Generative Adversarial Networks》
          (\href{https://papers.nips.cc/paper/5423-generative-adversarial-nets.pdf}{链接})所述,
          生成高质量图像,但训练不稳定。
\end{itemize}

\subsubsection{综合分析}
神经网络的多样性使其能够适应各种任务,从简单的FNN到复杂的Transformer,每种结构都有其独特优势。
选择合适结构需考虑数据类型、任务复杂性和计算资源。根据UpGrad的《Neural Network Architecture: 
Types, Components \& Key Algorithms》
(\href{https://www.upgrad.com/blog/neural-network-architecture-components-algorithms/}{链接}),
未来的研究可能进一步优化轻量级网络(如MobileNet)以适应移动设备。

\subsubsection{关键引用}
\begin{itemize}
    \item GeeksforGeeks Neural Networks Beginner's Guide: \\
          \url{https://www.geeksforgeeks.org/neural-networks-a-beginners-guide/}
    \item V7Labs Essential Guide to Neural Network Architectures: \\
          \url{https://www.v7labs.com/blog/neural-network-architectures-guide/}
    \item Wikipedia Neural Network Machine Learning: \\
          \url{https://en.wikipedia.org/wiki/Neural_network_(machine_learning)}
    \item MyGreatLearning Types of Neural Networks Definition: \\
          \url{https://www.mygreatlearning.com/blog/types-of-neural-networks/}
    \item UpGrad Neural Network Architecture Components Algorithms: \\
          \url{https://www.upgrad.com/blog/neural-network-architecture-components-algorithms/}
    \item V7Labs Convolutional Neural Networks Guide: \\
          \url{https://www.v7labs.com/blog/convolutional-neural-networks-guide/}
    \item V7Labs Recurrent Neural Networks Guide: \\
          \url{https://www.v7labs.com/blog/recurrent-neural-networks-guide/}
    \item Attention Is All You Need Transformer Paper: \\
          \url{https://arxiv.org/abs/1706.03762}
    \item Generative Adversarial Networks NIPS Paper: \\
          \url{https://papers.nips.cc/paper/5423-generative-adversarial-nets.pdf}
\end{itemize}

\cleardoublepage

\subsection{ResNet 与 Vision Transformer (ViT) 的结构对比}


\setlength\LTleft{0pt}   % 让 longtable 左对齐于正文左边界
\setlength\LTright{0pt}  % 让 longtable 右对齐于正文右边界
\begin{longtable}{@{}p{0.28\textwidth}p{0.34\textwidth}p{0.34\textwidth}@{}}
\caption{ResNet 与 Vision Transformer (ViT) 的详细结构对比} \\ 
\toprule
\textbf{比较维度} & \textbf{ResNet} & \textbf{Vision Transformer (ViT)} \\
\midrule
\endfirsthead

\toprule
\textbf{比较维度} & \textbf{ResNet} & \textbf{Vision Transformer (ViT)} \\
\midrule
\endhead

\textbf{架构类型}           & 卷积神经网络(CNN)                           & Transformer 架构                              \\
\textbf{提出年份}           & 2015                                         & 2020                                        \\
\textbf{提出机构}           & 微软研究院                                    & Google Brain                                \\
\textbf{基本单元}           & 卷积层 + 残差连接(Residual Block)            & 自注意力模块(Multi-head Attention) + MLP   \\
\textbf{参数量(Base模型)} & 较少(如 ResNet-50 约 25M)                   & 较多(如 ViT-B 约 86M)                     \\
\textbf{计算复杂度}         & 较低,主要是卷积操作                           & 高,自注意力为 \(O(n^2)\) 时间复杂度         \\
\textbf{输入处理}           & 原始图像直接进入卷积网络                       & 图像切成 Patch,再投影为序列                 \\
\textbf{位置建模方式}       & 隐式建模(卷积天然包含位置信息)               & 显式位置编码(Positional Encoding)         \\
\textbf{空间建模能力}       & 局部为主,靠堆叠层数扩大全局感受野             & 全局建模能力强(自注意力机制)               \\
\textbf{可解释性}           & 较强,可通过卷积特征图分析                     & 较弱,注意力机制不易解释                     \\
\textbf{收敛速度}           & 快速,适合从头训练                             & 慢,对初始化敏感                             \\
\textbf{是否需要预训练}     & 可以从头训练,也支持预训练                     & 强烈依赖预训练(无预训练效果差)             \\
\textbf{数据规模依赖}       & 中小规模数据也能表现良好                       & 需要大规模数据(如 ImageNet-21k)           \\
\textbf{训练资源需求}       & 普通 GPU 即可训练(如单卡)                    & 需多卡/TPU,大内存显卡更佳                  \\
\textbf{推理速度}           & 快(卷积并行度高)                             & 慢(序列操作限制并行度)                     \\
\textbf{适合任务}           & 图像分类、目标检测、语义分割等经典视觉任务     & 大规模视觉任务、跨模态学习、多任务联合建模   \\
\textbf{代表模型}           & ResNet-18/34/50/101/152                        & ViT-B/16, ViT-L/32, DeiT, Swin Transformer    \\
\bottomrule
\end{longtable}






\section{神经网络CNN训练(纯手写第一版)}

\subsection{CNN训练代码一}
\begin{lstlisting}[language=python, caption={神经网络CNN训练(纯手写第一版)}, label={lst:cnn_train_handwritten}]
import torch
import torch.nn as nn
import torch.optim as optim
import torchvision
import torchvision.transforms as transforms
from torch.utils.data import DataLoader, SubsetRandomSampler
import numpy as np

# 数据预处理 - 增强数据增强
transform_train = transforms.Compose([
    transforms.RandomCrop(32, padding=4),
    transforms.RandomHorizontalFlip(),
    transforms.ToTensor(),
    transforms.Normalize((0.4914, 0.4822, 0.4465), (0.2470, 0.2435, 0.2616))
])

transform_test = transforms.Compose([
    transforms.ToTensor(),
    transforms.Normalize((0.4914, 0.4822, 0.4465), (0.2470, 0.2435, 0.2616))
])

# 加载 CIFAR-10 数据集
trainset = torchvision.datasets.CIFAR10(root='./data', train=True, download=True, transform=transform_train)
testset = torchvision.datasets.CIFAR10(root='./data', train=False, download=True, transform=transform_test)

# 划分训练集和验证集
validation_split = 0.2  # 20% 用于验证集
dataset_size = len(trainset)
indices = list(range(dataset_size))
np.random.seed(42)  # 固定随机种子以确保可重复性
np.random.shuffle(indices)
split = int(np.floor(validation_split * dataset_size))
train_indices, val_indices = indices[split:], indices[:split]

# 创建 DataLoader
train_sampler = SubsetRandomSampler(train_indices)
val_sampler = SubsetRandomSampler(val_indices)

trainloader = DataLoader(trainset, batch_size=128, sampler=train_sampler, num_workers=2)
valloader = DataLoader(trainset, batch_size=128, sampler=val_sampler, num_workers=2)
testloader = DataLoader(testset, batch_size=128, shuffle=False, num_workers=2)

# 定义改进的 CNN 模型
class ImprovedCNN(nn.Module):
    def __init__(self):
        super(ImprovedCNN, self).__init__()
        
        # 第一个卷积块
        self.conv1 = nn.Sequential(
            nn.Conv2d(3, 64, 3, padding=1),
            nn.BatchNorm2d(64),
            nn.ReLU(inplace=True),
            nn.Conv2d(64, 64, 3, padding=1),
            nn.BatchNorm2d(64),
            nn.ReLU(inplace=True),
            nn.MaxPool2d(2, 2)
        )
        
        # 第二个卷积块
        self.conv2 = nn.Sequential(
            nn.Conv2d(64, 128, 3, padding=1),
            nn.BatchNorm2d(128),
            nn.ReLU(inplace=True),
            nn.Conv2d(128, 128, 3, padding=1),
            nn.BatchNorm2d(128),
            nn.ReLU(inplace=True),
            nn.MaxPool2d(2, 2)
        )
        
        # 第三个卷积块
        self.conv3 = nn.Sequential(
            nn.Conv2d(128, 256, 3, padding=1),
            nn.BatchNorm2d(256),
            nn.ReLU(inplace=True),
            nn.Conv2d(256, 256, 3, padding=1),
            nn.BatchNorm2d(256),
            nn.ReLU(inplace=True),
            nn.MaxPool2d(2, 2)
        )
        
        # 全连接层
        self.fc = nn.Sequential(
            nn.Dropout(0.5),
            nn.Linear(256 * 4 * 4, 512),
            nn.BatchNorm1d(512),
            nn.ReLU(inplace=True),
            nn.Dropout(0.5),
            nn.Linear(512, 10)
        )
        
    def forward(self, x):
        x = self.conv1(x)
        x = self.conv2(x)
        x = self.conv3(x)
        x = x.view(x.size(0), -1)
        x = self.fc(x)
        return x

# 主程序
if __name__ == '__main__':
    # 实例化模型
    model = ImprovedCNN()

    # 定义损失函数和优化器
    criterion = nn.CrossEntropyLoss()
    optimizer = optim.SGD(model.parameters(), lr=0.01, momentum=0.9, weight_decay=5e-4)
    # 学习率调度器
    scheduler = optim.lr_scheduler.ReduceLROnPlateau(optimizer, 'min', factor=0.1, patience=5, verbose=True)

    # 如果有 GPU,将模型移动到 GPU
    device = torch.device('cuda' if torch.cuda.is_available() else 'cpu')
    model.to(device)
    print(f"使用设备: {device}")

    # 训练模型
    best_val_acc = 0.0
    for epoch in range(100):
        model.train()
        running_loss = 0.0
        correct = 0
        total = 0
        
        for i, (inputs, labels) in enumerate(trainloader):
            inputs, labels = inputs.to(device), labels.to(device)
            
            optimizer.zero_grad()
            outputs = model(inputs)
            loss = criterion(outputs, labels)
            loss.backward()
            optimizer.step()
            
            running_loss += loss.item()
            _, predicted = outputs.max(1)
            total += labels.size(0)
            correct += predicted.eq(labels).sum().item()
            
            if i % 100 == 99:
                print(f'[{epoch + 1}, {i + 1}] loss: {running_loss / 100:.3f} | acc: {100.*correct/total:.2f}%')
                running_loss = 0.0
        
        # 每个epoch结束后验证
        model.eval()
        val_loss = 0
        correct = 0
        total = 0
        with torch.no_grad():
            for data in valloader:
                images, labels = data
                images, labels = images.to(device), labels.to(device)
                outputs = model(images)
                loss = criterion(outputs, labels)
                val_loss += loss.item()
                _, predicted = torch.max(outputs.data, 1)
                total += labels.size(0)
                correct += (predicted == labels).sum().item()
        
        val_acc = 100. * correct / total
        print(f'Epoch {epoch+1}: 验证准确率: {val_acc:.2f}%')
        
        # 更新学习率
        scheduler.step(val_loss)
        
        # 保存验证集上最佳模型
        if val_acc > best_val_acc:
            best_val_acc = val_acc
            torch.save(model.state_dict(), 'best_cifar10_model.pth')
            print(f'保存最佳模型,验证准确率: {best_val_acc:.2f}%')

    print('训练完成')

    # 加载最佳模型进行测试
    model.load_state_dict(torch.load('best_cifar10_model.pth'))
    model.eval()
    correct = 0
    total = 0
    with torch.no_grad():
        for data in testloader:
            images, labels = data
            images, labels = images.to(device), labels.to(device)
            outputs = model(images)
            _, predicted = torch.max(outputs.data, 1)
            total += labels.size(0)
            correct += (predicted == labels).sum().item()

    print(f'最佳模型在测试集上的准确率: {100 * correct / total:.2f}%')
\end{lstlisting}


\subsection{CNN训练结果一}

\begin{lstlisting}[language=python, caption={CNN训练结果一}, label={lst:cnn_train_result}]
PS C:\Users\78003\Desktop\prml_re> conda activate yclearning
PS C:\Users\78003\Desktop\prml_re> python cifar10_classifier.py

使用设备: cuda
[1, 100] loss: 1.809 | acc: 33.21%
[1, 200] loss: 1.504 | acc: 38.73%
[1, 300] loss: 1.352 | acc: 42.79%
Epoch 1: 验证准确率: 56.47%
保存最佳模型,验证准确率: 56.47%
[2, 100] loss: 1.199 | acc: 56.58%
[2, 200] loss: 1.103 | acc: 58.38%
[2, 300] loss: 1.011 | acc: 60.15%
Epoch 2: 验证准确率: 64.90%
保存最佳模型,验证准确率: 64.90%
[3, 100] loss: 0.954 | acc: 66.12%
[3, 200] loss: 0.917 | acc: 66.95%
[3, 300] loss: 0.871 | acc: 67.58%
Epoch 3: 验证准确率: 70.78%
保存最佳模型,验证准确率: 70.78%
[4, 100] loss: 0.831 | acc: 70.93%
[4, 200] loss: 0.798 | acc: 71.69%
[4, 300] loss: 0.782 | acc: 71.95%
Epoch 4: 验证准确率: 73.30%
保存最佳模型,验证准确率: 73.30%
[5, 100] loss: 0.739 | acc: 73.97%
[5, 200] loss: 0.731 | acc: 74.01%
[5, 300] loss: 0.714 | acc: 74.43%
Epoch 5: 验证准确率: 76.20%
保存最佳模型,验证准确率: 76.20%
[6, 100] loss: 0.680 | acc: 76.30%
[6, 200] loss: 0.683 | acc: 76.29%
[6, 300] loss: 0.668 | acc: 76.58%
Epoch 6: 验证准确率: 77.85%
保存最佳模型,验证准确率: 77.85%
[7, 100] loss: 0.635 | acc: 77.83%
[7, 200] loss: 0.629 | acc: 78.08%
[7, 300] loss: 0.614 | acc: 78.16%
Epoch 7: 验证准确率: 77.90%
保存最佳模型,验证准确率: 77.90%
[8, 100] loss: 0.570 | acc: 80.14%
[8, 200] loss: 0.592 | acc: 79.82%
[8, 300] loss: 0.578 | acc: 79.85%
Epoch 8: 验证准确率: 74.53%
[9, 100] loss: 0.560 | acc: 80.34%
[9, 200] loss: 0.539 | acc: 80.91%
[9, 300] loss: 0.563 | acc: 80.74%
Epoch 9: 验证准确率: 79.64%
保存最佳模型,验证准确率: 79.64%
[10, 100] loss: 0.531 | acc: 81.48%
[10, 200] loss: 0.518 | acc: 81.74%
[10, 300] loss: 0.520 | acc: 81.86%
Epoch 10: 验证准确率: 79.66%
保存最佳模型,验证准确率: 79.66%
[11, 100] loss: 0.496 | acc: 82.55%
[11, 200] loss: 0.489 | acc: 82.66%
[11, 300] loss: 0.508 | acc: 82.54%
Epoch 11: 验证准确率: 81.02%
保存最佳模型,验证准确率: 81.02%
[12, 100] loss: 0.476 | acc: 83.37%
[12, 200] loss: 0.473 | acc: 83.50%
[12, 300] loss: 0.481 | acc: 83.46%
Epoch 12: 验证准确率: 81.05%
保存最佳模型,验证准确率: 81.05%
[13, 100] loss: 0.444 | acc: 84.39%
[13, 200] loss: 0.450 | acc: 84.26%
[13, 300] loss: 0.462 | acc: 84.24%
Epoch 13: 验证准确率: 82.22%
保存最佳模型,验证准确率: 82.22%
[14, 100] loss: 0.439 | acc: 85.30%
[14, 200] loss: 0.439 | acc: 84.95%
[14, 300] loss: 0.431 | acc: 85.12%
Epoch 14: 验证准确率: 82.90%
保存最佳模型,验证准确率: 82.90%
[15, 100] loss: 0.404 | acc: 86.21%
[15, 200] loss: 0.430 | acc: 85.62%
[15, 300] loss: 0.428 | acc: 85.62%
Epoch 15: 验证准确率: 82.29%
[16, 100] loss: 0.404 | acc: 86.64%
[16, 200] loss: 0.399 | acc: 86.36%
[16, 300] loss: 0.410 | acc: 86.12%
Epoch 16: 验证准确率: 81.61%
[17, 100] loss: 0.367 | acc: 87.30%
[17, 200] loss: 0.396 | acc: 86.84%
[17, 300] loss: 0.394 | acc: 86.74%
Epoch 17: 验证准确率: 84.65%
保存最佳模型,验证准确率: 84.65%
[18, 100] loss: 0.378 | acc: 86.97%
[18, 200] loss: 0.361 | acc: 87.34%
[18, 300] loss: 0.385 | acc: 87.26%
Epoch 18: 验证准确率: 84.91%
保存最佳模型,验证准确率: 84.91%
[19, 100] loss: 0.355 | acc: 88.05%
[19, 200] loss: 0.356 | acc: 87.84%
[19, 300] loss: 0.356 | acc: 87.79%
Epoch 19: 验证准确率: 84.83%
[20, 100] loss: 0.349 | acc: 87.92%
[20, 200] loss: 0.351 | acc: 87.86%
[20, 300] loss: 0.345 | acc: 87.98%
Epoch 20: 验证准确率: 83.97%
[21, 100] loss: 0.327 | acc: 88.64%
[21, 200] loss: 0.334 | acc: 88.61%
[21, 300] loss: 0.350 | acc: 88.48%
Epoch 21: 验证准确率: 85.73%
保存最佳模型,验证准确率: 85.73%
[22, 100] loss: 0.314 | acc: 89.00%
[22, 200] loss: 0.327 | acc: 88.87%
[22, 300] loss: 0.329 | acc: 88.83%
Epoch 22: 验证准确率: 86.06%
保存最佳模型,验证准确率: 86.06%
[23, 100] loss: 0.303 | acc: 89.61%
[23, 200] loss: 0.317 | acc: 89.28%
[23, 300] loss: 0.313 | acc: 89.32%
Epoch 23: 验证准确率: 84.69%
[24, 100] loss: 0.296 | acc: 89.80%
[24, 200] loss: 0.296 | acc: 89.88%
[24, 300] loss: 0.305 | acc: 89.72%
Epoch 24: 验证准确率: 86.34%
保存最佳模型,验证准确率: 86.34%
[25, 100] loss: 0.276 | acc: 90.61%
[25, 200] loss: 0.299 | acc: 90.16%
[25, 300] loss: 0.295 | acc: 90.01%
Epoch 25: 验证准确率: 85.66%
[26, 100] loss: 0.279 | acc: 90.29%
[26, 200] loss: 0.284 | acc: 90.25%
[26, 300] loss: 0.290 | acc: 90.18%
Epoch 26: 验证准确率: 85.39%
[27, 100] loss: 0.264 | acc: 90.84%
[27, 200] loss: 0.276 | acc: 90.66%
[27, 300] loss: 0.285 | acc: 90.42%
Epoch 27: 验证准确率: 86.06%
[28, 100] loss: 0.259 | acc: 91.16%
[28, 200] loss: 0.274 | acc: 90.86%
[28, 300] loss: 0.273 | acc: 90.73%
Epoch 28: 验证准确率: 87.06%
保存最佳模型,验证准确率: 87.06%
[29, 100] loss: 0.252 | acc: 91.41%
[29, 200] loss: 0.251 | acc: 91.33%
[29, 300] loss: 0.276 | acc: 91.04%
Epoch 29: 验证准确率: 85.27%
[30, 100] loss: 0.247 | acc: 91.32%
[30, 200] loss: 0.248 | acc: 91.41%
[30, 300] loss: 0.254 | acc: 91.36%
Epoch 30: 验证准确率: 85.89%
[31, 100] loss: 0.231 | acc: 91.73%
[31, 200] loss: 0.241 | acc: 91.70%
[31, 300] loss: 0.260 | acc: 91.41%
Epoch 31: 验证准确率: 86.26%
[32, 100] loss: 0.225 | acc: 92.10%
[32, 200] loss: 0.241 | acc: 91.93%
[32, 300] loss: 0.239 | acc: 91.88%
Epoch 32: 验证准确率: 85.79%
[33, 100] loss: 0.223 | acc: 92.39%
[33, 200] loss: 0.227 | acc: 92.32%
[33, 300] loss: 0.240 | acc: 92.12%
Epoch 33: 验证准确率: 86.09%
[34, 100] loss: 0.216 | acc: 92.52%
[34, 200] loss: 0.224 | acc: 92.39%
[34, 300] loss: 0.224 | acc: 92.26%
Epoch 34: 验证准确率: 86.81%
[35, 100] loss: 0.174 | acc: 94.02%
[35, 200] loss: 0.163 | acc: 94.31%
[35, 300] loss: 0.153 | acc: 94.62%
Epoch 35: 验证准确率: 89.76%
保存最佳模型,验证准确率: 89.76%
[36, 100] loss: 0.137 | acc: 95.43%
[36, 200] loss: 0.143 | acc: 95.30%
[36, 300] loss: 0.147 | acc: 95.28%
Epoch 36: 验证准确率: 90.03%
保存最佳模型,验证准确率: 90.03%
[37, 100] loss: 0.127 | acc: 95.79%
[37, 200] loss: 0.133 | acc: 95.71%
[37, 300] loss: 0.134 | acc: 95.64%
Epoch 37: 验证准确率: 89.60%
[38, 100] loss: 0.128 | acc: 95.79%
[38, 200] loss: 0.130 | acc: 95.79%
[38, 300] loss: 0.127 | acc: 95.82%
Epoch 38: 验证准确率: 90.13%
保存最佳模型,验证准确率: 90.13%
[39, 100] loss: 0.116 | acc: 96.22%
[39, 200] loss: 0.121 | acc: 96.20%
[39, 300] loss: 0.125 | acc: 96.17%
Epoch 39: 验证准确率: 90.29%
保存最佳模型,验证准确率: 90.29%
[40, 100] loss: 0.113 | acc: 96.48%
[40, 200] loss: 0.115 | acc: 96.36%
[40, 300] loss: 0.120 | acc: 96.22%
Epoch 40: 验证准确率: 90.32%
保存最佳模型,验证准确率: 90.32%
[41, 100] loss: 0.113 | acc: 96.33%
[41, 200] loss: 0.112 | acc: 96.38%
[41, 300] loss: 0.110 | acc: 96.39%
Epoch 41: 验证准确率: 90.15%
[42, 100] loss: 0.108 | acc: 96.53%
[42, 200] loss: 0.119 | acc: 96.36%
[42, 300] loss: 0.111 | acc: 96.41%
Epoch 42: 验证准确率: 90.12%
[43, 100] loss: 0.103 | acc: 96.80%
[43, 200] loss: 0.113 | acc: 96.61%
[43, 300] loss: 0.114 | acc: 96.49%
Epoch 43: 验证准确率: 90.38%
保存最佳模型,验证准确率: 90.38%
[44, 100] loss: 0.108 | acc: 96.48%
[44, 200] loss: 0.102 | acc: 96.60%
[44, 300] loss: 0.111 | acc: 96.54%
Epoch 44: 验证准确率: 90.61%
保存最佳模型,验证准确率: 90.61%
[45, 100] loss: 0.100 | acc: 96.78%
[45, 200] loss: 0.106 | acc: 96.77%
[45, 300] loss: 0.101 | acc: 96.79%
Epoch 45: 验证准确率: 90.46%
[46, 100] loss: 0.099 | acc: 96.74%
[46, 200] loss: 0.105 | acc: 96.65%
[46, 300] loss: 0.101 | acc: 96.71%
Epoch 46: 验证准确率: 90.20%
[47, 100] loss: 0.096 | acc: 96.93%
[47, 200] loss: 0.102 | acc: 96.77%
[47, 300] loss: 0.100 | acc: 96.77%
Epoch 47: 验证准确率: 90.17%
[48, 100] loss: 0.101 | acc: 96.80%
[48, 200] loss: 0.097 | acc: 96.82%
[48, 300] loss: 0.099 | acc: 96.81%
Epoch 48: 验证准确率: 90.44%
[49, 100] loss: 0.094 | acc: 96.95%
[49, 200] loss: 0.090 | acc: 97.04%
[49, 300] loss: 0.091 | acc: 97.10%
Epoch 49: 验证准确率: 90.40%
[50, 100] loss: 0.089 | acc: 97.23%
[50, 200] loss: 0.092 | acc: 97.12%
[50, 300] loss: 0.090 | acc: 97.15%
Epoch 50: 验证准确率: 90.52%
[51, 100] loss: 0.087 | acc: 97.48%
[51, 200] loss: 0.087 | acc: 97.37%
[51, 300] loss: 0.093 | acc: 97.24%
Epoch 51: 验证准确率: 90.48%
[52, 100] loss: 0.088 | acc: 97.29%
[52, 200] loss: 0.089 | acc: 97.29%
[52, 300] loss: 0.086 | acc: 97.26%
Epoch 52: 验证准确率: 90.55%
[53, 100] loss: 0.090 | acc: 97.21%
[53, 200] loss: 0.089 | acc: 97.25%
[53, 300] loss: 0.087 | acc: 97.23%
Epoch 53: 验证准确率: 90.48%
[54, 100] loss: 0.091 | acc: 97.01%
[54, 200] loss: 0.088 | acc: 97.12%
[54, 300] loss: 0.087 | acc: 97.20%
Epoch 54: 验证准确率: 90.57%
[55, 100] loss: 0.088 | acc: 97.13%
[55, 200] loss: 0.088 | acc: 97.15%
[55, 300] loss: 0.087 | acc: 97.16%
Epoch 55: 验证准确率: 90.77%
保存最佳模型,验证准确率: 90.77%
[56, 100] loss: 0.086 | acc: 97.30%
[56, 200] loss: 0.084 | acc: 97.47%
[56, 300] loss: 0.083 | acc: 97.46%
Epoch 56: 验证准确率: 90.63%
[57, 100] loss: 0.089 | acc: 97.21%
[57, 200] loss: 0.086 | acc: 97.24%
[57, 300] loss: 0.087 | acc: 97.18%
Epoch 57: 验证准确率: 90.71%
[58, 100] loss: 0.086 | acc: 97.09%
[58, 200] loss: 0.083 | acc: 97.28%
[58, 300] loss: 0.085 | acc: 97.27%
Epoch 58: 验证准确率: 90.69%
[59, 100] loss: 0.083 | acc: 97.47%
[59, 200] loss: 0.077 | acc: 97.57%
[59, 300] loss: 0.089 | acc: 97.42%
Epoch 59: 验证准确率: 90.40%
[60, 100] loss: 0.081 | acc: 97.55%
[60, 200] loss: 0.085 | acc: 97.43%
[60, 300] loss: 0.084 | acc: 97.40%
Epoch 60: 验证准确率: 90.52%
[61, 100] loss: 0.084 | acc: 97.29%
[61, 200] loss: 0.080 | acc: 97.38%
[61, 300] loss: 0.082 | acc: 97.40%
Epoch 61: 验证准确率: 90.67%
[62, 100] loss: 0.084 | acc: 97.35%
[62, 200] loss: 0.084 | acc: 97.37%
[62, 300] loss: 0.088 | acc: 97.34%
Epoch 62: 验证准确率: 90.44%
[63, 100] loss: 0.090 | acc: 97.15%
[63, 200] loss: 0.079 | acc: 97.42%
[63, 300] loss: 0.085 | acc: 97.39%
Epoch 63: 验证准确率: 90.39%
[64, 100] loss: 0.083 | acc: 97.39%
[64, 200] loss: 0.086 | acc: 97.25%
[64, 300] loss: 0.084 | acc: 97.24%
Epoch 64: 验证准确率: 90.70%
[65, 100] loss: 0.093 | acc: 97.04%
[65, 200] loss: 0.086 | acc: 97.08%
[65, 300] loss: 0.081 | acc: 97.21%
Epoch 65: 验证准确率: 90.24%
[66, 100] loss: 0.088 | acc: 97.16%
[66, 200] loss: 0.082 | acc: 97.29%
[66, 300] loss: 0.083 | acc: 97.32%
Epoch 66: 验证准确率: 90.48%
[67, 100] loss: 0.084 | acc: 97.45%
[67, 200] loss: 0.086 | acc: 97.33%
[67, 300] loss: 0.078 | acc: 97.41%
Epoch 67: 验证准确率: 90.39%
[68, 100] loss: 0.083 | acc: 97.55%
[68, 200] loss: 0.085 | acc: 97.44%
[68, 300] loss: 0.079 | acc: 97.50%
Epoch 68: 验证准确率: 90.56%
[69, 100] loss: 0.082 | acc: 97.55%
[69, 200] loss: 0.085 | acc: 97.42%
[69, 300] loss: 0.083 | acc: 97.43%
Epoch 69: 验证准确率: 90.32%
[70, 100] loss: 0.084 | acc: 97.33%
[70, 200] loss: 0.087 | acc: 97.27%
[70, 300] loss: 0.088 | acc: 97.21%
Epoch 70: 验证准确率: 90.07%
[71, 100] loss: 0.089 | acc: 97.13%
[71, 200] loss: 0.085 | acc: 97.19%
[71, 300] loss: 0.089 | acc: 97.18%
Epoch 71: 验证准确率: 90.63%
[72, 100] loss: 0.085 | acc: 97.14%
[72, 200] loss: 0.082 | acc: 97.31%
[72, 300] loss: 0.084 | acc: 97.33%
Epoch 72: 验证准确率: 90.39%
[73, 100] loss: 0.082 | acc: 97.50%
[73, 200] loss: 0.085 | acc: 97.50%
[73, 300] loss: 0.086 | acc: 97.45%
Epoch 73: 验证准确率: 90.45%
[74, 100] loss: 0.083 | acc: 97.33%
[74, 200] loss: 0.086 | acc: 97.33%
[74, 300] loss: 0.082 | acc: 97.33%
Epoch 74: 验证准确率: 90.37%
[75, 100] loss: 0.083 | acc: 97.50%
[75, 200] loss: 0.085 | acc: 97.44%
[75, 300] loss: 0.086 | acc: 97.33%
Epoch 75: 验证准确率: 90.35%
[76, 100] loss: 0.083 | acc: 97.48%
[76, 200] loss: 0.080 | acc: 97.46%
[76, 300] loss: 0.089 | acc: 97.41%
Epoch 76: 验证准确率: 90.45%
[77, 100] loss: 0.079 | acc: 97.59%
[77, 200] loss: 0.085 | acc: 97.39%
[77, 300] loss: 0.088 | acc: 97.33%
Epoch 77: 验证准确率: 90.40%
[78, 100] loss: 0.079 | acc: 97.53%
[78, 200] loss: 0.082 | acc: 97.51%
[78, 300] loss: 0.091 | acc: 97.34%
Epoch 78: 验证准确率: 90.32%
[79, 100] loss: 0.084 | acc: 97.15%
[79, 200] loss: 0.084 | acc: 97.23%
[79, 300] loss: 0.081 | acc: 97.31%
Epoch 79: 验证准确率: 90.33%
[80, 100] loss: 0.088 | acc: 97.33%
[80, 200] loss: 0.083 | acc: 97.33%
[80, 300] loss: 0.080 | acc: 97.46%
Epoch 80: 验证准确率: 90.48%
[81, 100] loss: 0.085 | acc: 97.38%
[81, 200] loss: 0.084 | acc: 97.32%
[81, 300] loss: 0.085 | acc: 97.24%
Epoch 81: 验证准确率: 90.67%
[82, 100] loss: 0.081 | acc: 97.54%
[82, 200] loss: 0.082 | acc: 97.45%
[82, 300] loss: 0.082 | acc: 97.48%
Epoch 82: 验证准确率: 90.43%
[83, 100] loss: 0.079 | acc: 97.72%
[83, 200] loss: 0.082 | acc: 97.52%
[83, 300] loss: 0.087 | acc: 97.45%
Epoch 83: 验证准确率: 90.40%
[84, 100] loss: 0.083 | acc: 97.50%
[84, 200] loss: 0.082 | acc: 97.45%
[84, 300] loss: 0.084 | acc: 97.41%
Epoch 84: 验证准确率: 90.07%
[85, 100] loss: 0.083 | acc: 97.30%
[85, 200] loss: 0.084 | acc: 97.27%
[85, 300] loss: 0.085 | acc: 97.25%
Epoch 85: 验证准确率: 90.24%
[86, 100] loss: 0.085 | acc: 97.32%
[86, 200] loss: 0.081 | acc: 97.38%
[86, 300] loss: 0.083 | acc: 97.38%
Epoch 86: 验证准确率: 90.34%
[87, 100] loss: 0.087 | acc: 97.26%
[87, 200] loss: 0.085 | acc: 97.33%
[87, 300] loss: 0.088 | acc: 97.31%
Epoch 87: 验证准确率: 90.45%
[88, 100] loss: 0.085 | acc: 97.38%
[88, 200] loss: 0.084 | acc: 97.32%
[88, 300] loss: 0.085 | acc: 97.33%
Epoch 88: 验证准确率: 90.38%
[89, 100] loss: 0.084 | acc: 97.45%
[89, 200] loss: 0.080 | acc: 97.48%
[89, 300] loss: 0.085 | acc: 97.39%
Epoch 89: 验证准确率: 90.62%
[90, 100] loss: 0.087 | acc: 97.17%
[90, 200] loss: 0.085 | acc: 97.19%
[90, 300] loss: 0.082 | acc: 97.32%
Epoch 90: 验证准确率: 90.41%
[91, 100] loss: 0.085 | acc: 97.27%
[91, 200] loss: 0.086 | acc: 97.31%
[91, 300] loss: 0.086 | acc: 97.30%
Epoch 91: 验证准确率: 90.59%
[92, 100] loss: 0.085 | acc: 97.30%
[92, 200] loss: 0.087 | acc: 97.32%
[92, 300] loss: 0.082 | acc: 97.39%
Epoch 92: 验证准确率: 90.71%
[93, 100] loss: 0.089 | acc: 97.17%
[93, 200] loss: 0.085 | acc: 97.23%
[93, 300] loss: 0.086 | acc: 97.25%
Epoch 93: 验证准确率: 90.59%
[94, 100] loss: 0.084 | acc: 97.45%
[94, 200] loss: 0.088 | acc: 97.39%
[94, 300] loss: 0.087 | acc: 97.31%
Epoch 94: 验证准确率: 90.52%
[95, 100] loss: 0.089 | acc: 97.12%
[95, 200] loss: 0.083 | acc: 97.21%
[95, 300] loss: 0.083 | acc: 97.21%
Epoch 95: 验证准确率: 90.66%
[96, 100] loss: 0.085 | acc: 97.41%
[96, 200] loss: 0.082 | acc: 97.48%
[96, 300] loss: 0.080 | acc: 97.49%
Epoch 96: 验证准确率: 90.63%
[97, 100] loss: 0.086 | acc: 97.29%
[97, 200] loss: 0.081 | acc: 97.41%
[97, 300] loss: 0.088 | acc: 97.37%
Epoch 97: 验证准确率: 90.34%
[98, 100] loss: 0.083 | acc: 97.54%
[98, 200] loss: 0.086 | acc: 97.45%
[98, 300] loss: 0.079 | acc: 97.49%
Epoch 98: 验证准确率: 90.28%
[99, 100] loss: 0.081 | acc: 97.44%
[99, 200] loss: 0.087 | acc: 97.30%
[99, 300] loss: 0.084 | acc: 97.33%
Epoch 99: 验证准确率: 90.43%
[100, 100] loss: 0.083 | acc: 97.47%
[100, 200] loss: 0.084 | acc: 97.43%
[100, 300] loss: 0.085 | acc: 97.39%
Epoch 100: 验证准确率: 90.59%
训练完成
最佳模型在测试集上的准确率: 90.60%
\end{lstlisting}














% 附录 A
\chapter*{附录 A. 中英文对照表}\addcontentsline{toc}{chapter}{附录 A. 中英文对照表}   
\thispagestyle{plain} 
\setcounter{section}{0}   
\renewcommand\thesection{A.\arabic{section}}   
\renewcommand{\thefigure}{A.\arabic{figure}} 
\renewcommand{\thetable}{A.\arabic{table}}

\section{中英文对照表}
\begin{multicols}{2}  

\begin{table}[H]
\centering
\caption{\textbf{中英文对照表}}
\begin{tabular}{ll}
\toprule
英文 & 中文 \\
\midrule
Bayes classification & 贝叶斯分类 \\
decision rule & 决策规则 \\
minimum error rate & 最小错误率 \\
minimum risk & 最小风险 \\
rejection option & 拒识选项 \\
Gaussian distribution & 高斯分布 \\
covariance matrix & 协方差矩阵 \\
discriminant function & 判别函数 \\
decision boundary & 决策边界 \\
Hidden Markov Model & 隐马尔可夫模型 \\
hidden state & 隐状态 \\
observation sequence & 观测序列 \\
maximum likelihood estimation & 最大似然估计 \\
Bayesian estimation & 贝叶斯估计 \\
k-Nearest Neighbor & k近邻 \\
Parzen window & Parzen窗 \\
linear discriminant & 线性判别 \\
quadratic discriminant & 二次判别 \\
prior probability & 先验概率 \\
posterior probability & 后验概率 \\
\bottomrule
\end{tabular}
\end{table}

\end{multicols}






\end{document}