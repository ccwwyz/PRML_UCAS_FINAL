

% 若编译失败,且生成 .synctex(busy) 辅助文件,可能有两个原因:
% 1. 需要插入的图片不存在:Ctrl + F 搜索 'figure' 将这些代码注释/删除掉即可
% 2. 路径/文件名含中文或空格:更改路径/文件名即可

% ------------------------------------------------------------- %
% >> ------------------ 文章宏包及相关设置 ------------------ << %
% 设定文章类型与编码格式
\documentclass[UTF8]{report}		

% 本文特殊宏包
\usepackage{siunitx} % 埃米单位

% 本 .tex 专属的宏定义
    \def\V{\ \mathrm{V}}
    \def\mV{\ \mathrm{mV}}
    \def\kV{\ \mathrm{KV}}
    \def\KV{\ \mathrm{KV}}
    \def\MV{\ \mathrm{MV}}
    \def\A{\ \mathrm{A}}
    \def\mA{\ \mathrm{mA}}
    \def\kA{\ \mathrm{KA}}
    \def\KA{\ \mathrm{KA}}
    \def\MA{\ \mathrm{MA}}
    \def\O{\ \Omega}
    \def\mO{\ \Omega}
    \def\kO{\ \mathrm{K}\Omega}
    \def\KO{\ \mathrm{K}\Omega}
    \def\MO{\ \mathrm{M}\Omega}
    \def\Hz{\ \mathrm{Hz}}

% 自定义宏定义
    \def\N{\mathbb{N}}
    \def\F{\mathbb{F}}
    \def\Z{\mathbb{Z}}
    \def\Q{\mathbb{Q}}
    \def\R{\mathbb{R}}
    \def\C{\mathbb{C}}
    \def\T{\mathbb{T}}
    \def\S{\mathbb{S}}
    \def\A{\mathbb{A}}
    \def\I{\mathscr{I}}
    \def\Im{\mathrm{Im\,}}
    \def\Re{\mathrm{Re\,}}
    \def\d{\mathrm{d}}
    \def\p{\partial}

% 导入基本宏包
    \usepackage[UTF8]{ctex}     % 设置文档为中文语言
    \usepackage[colorlinks, linkcolor=blue, anchorcolor=blue, citecolor=blue, urlcolor=blue]{hyperref}  % 宏包:自动生成超链接 (此宏包与标题中的数学环境冲突)
    % \usepackage{hyperref}  % 宏包:自动生成超链接 (此宏包与标题中的数学环境冲突)
    % \hypersetup{
    %     colorlinks=true,    % false:边框链接 ; true:彩色链接
    %     citecolor={blue},    % 文献引用颜色
    %     linkcolor={blue},   % 目录 (我们在目录处单独设置),公式,图表,脚注等内部链接颜色
    %     urlcolor={orange},    % 网页 URL 链接颜色,包括 \href 中的 text
    %     % cyan 浅蓝色 
    %     % magenta 洋红色
    %     % yellow 黄色
    %     % black 黑色
    %     % white 白色
    %     % red 红色
    %     % green 绿色
    %     % blue 蓝色
    %     % gray 灰色
    %     % darkgray 深灰色
    %     % lightgray 浅灰色
    %     % brown 棕色
    %     % lime 石灰色
    %     % olive 橄榄色
    %     % orange 橙色
    %     % pink 粉红色
    %     % purple 紫色
    %     % teal 蓝绿色
    %     % violet 紫罗兰色
    % }

    % \usepackage{docmute}    % 宏包:子文件导入时自动去除导言区,用于主/子文件的写作方式,\include{./51单片机笔记}即可。注:启用此宏包会导致.tex文件capacity受限。
    \usepackage{amsmath}    % 宏包:数学公式
    \usepackage{mathrsfs}   % 宏包:提供更多数学符号
    \usepackage{amssymb}    % 宏包:提供更多数学符号
    \usepackage{pifont}     % 宏包:提供了特殊符号和字体
    \usepackage{extarrows}  % 宏包:更多箭头符号
    \usepackage{multicol}   % 宏包:支持多栏 
    \usepackage{graphicx}   % 宏包:插入图片
    \usepackage{float}      % 宏包:设置图片浮动位置
    %\usepackage{article}    % 宏包:使文本排版更加优美
    \usepackage{tikz}       % 宏包:绘图工具
    %\usepackage{pgfplots}   % 宏包:绘图工具
    \usepackage{enumerate}  % 宏包:列表环境设置
    \usepackage{enumitem}   % 宏包:列表环境设置

% 文章页面margin设置
    \usepackage[a4paper]{geometry}
        \geometry{top=1in}
        \geometry{bottom=1in}
        \geometry{left=0.75in}
        \geometry{right=0.75in}   % 设置上下左右页边距
        \geometry{marginparwidth=1.75cm}    % 设置边注距离(注释、标记等)

% 定义 solution 环境
\usepackage{amsthm}
\newtheorem{solution}{Solution}
        \geometry{bottom=1in}
        \geometry{left=0.75in}
        \geometry{right=0.75in}   % 设置上下左右页边距
        \geometry{marginparwidth=1.75cm}    % 设置边注距离(注释、标记等)

% 配置数学环境
    \usepackage{amsthm} % 宏包:数学环境配置
    % theorem-line 环境自定义
        \newtheoremstyle{MyLineTheoremStyle}% <name>
            {11pt}% <space above>
            {11pt}% <space below>
            {}% <body font> 使用默认正文字体
            {}% <indent amount>
            {\bfseries}% <theorem head font> 设置标题项为加粗
            {:}% <punctuation after theorem head>
            {.5em}% <space after theorem head>
            {\textbf{#1}\thmnumber{#2}\ \ (\,\textbf{#3}\,)}% 设置标题内容顺序
        \theoremstyle{MyLineTheoremStyle} % 应用自定义的定理样式
        \newtheorem{LineTheorem}{Theorem.\,}
    % theorem-block 环境自定义
        \newtheoremstyle{MyBlockTheoremStyle}% <name>
            {11pt}% <space above>
            {11pt}% <space below>
            {}% <body font> 使用默认正文字体
            {}% <indent amount>
            {\bfseries}% <theorem head font> 设置标题项为加粗
            {:\\ \indent}% <punctuation after theorem head>
            {.5em}% <space after theorem head>
            {\textbf{#1}\thmnumber{#2}\ \ (\,\textbf{#3}\,)}% 设置标题内容顺序
        \theoremstyle{MyBlockTheoremStyle} % 应用自定义的定理样式
        \newtheorem{BlockTheorem}[LineTheorem]{Theorem.\,} % 使用 LineTheorem 的计数器
    % definition 环境自定义
        \newtheoremstyle{MySubsubsectionStyle}% <name>
            {11pt}% <space above>
            {11pt}% <space below>
            {}% <body font> 使用默认正文字体
            {}% <indent amount>
            {\bfseries}% <theorem head font> 设置标题项为加粗
           % {:\\ \indent}% <punctuation after theorem head>
            {\\\indent}
            {0pt}% <space after theorem head>
            {\textbf{#3}}% 设置标题内容顺序
        \theoremstyle{MySubsubsectionStyle} % 应用自定义的定理样式
        \newtheorem{definition}{}

%宏包:有色文本框(proof环境)及其设置
    \usepackage[dvipsnames,svgnames]{xcolor}    %设置插入的文本框颜色
    \usepackage[strict]{changepage}     % 提供一个 adjustwidth 环境
    \usepackage{framed}     % 实现方框效果
        \definecolor{graybox_color}{rgb}{0.95,0.95,0.96} % 文本框颜色。修改此行中的 rgb 数值即可改变方框纹颜色,具体颜色的rgb数值可以在网站https://colordrop.io/ 中获得。(截止目前的尝试还没有成功过,感觉单位不一样)(找到喜欢的颜色,点击下方的小眼睛,找到rgb值,复制修改即可)
        \newenvironment{graybox}{%
        \def\FrameCommand{%
        \hspace{1pt}%
        {\color{gray}\small \vrule width 2pt}%
        {\color{graybox_color}\vrule width 4pt}%
        \colorbox{graybox_color}%
        }%
        \MakeFramed{\advance\hsize-\width\FrameRestore}%
        \noindent\hspace{-4.55pt}% disable indenting first paragraph
        \begin{adjustwidth}{}{7pt}%
        \vspace{2pt}\vspace{2pt}%
        }
        {%
        \vspace{2pt}\end{adjustwidth}\endMakeFramed%
        }



% 外源代码插入设置
    % matlab 代码插入设置
    \usepackage{matlab-prettifier}
        \lstset{style=Matlab-editor}    % 继承 matlab 代码高亮 , 此行不能删去
    \usepackage[most]{tcolorbox} % 引入tcolorbox包 
    \usepackage{listings} % 引入listings包
        \tcbuselibrary{listings, skins, breakable}
        \newfontfamily\codefont{Consolas} % 定义需要的 codefont 字体
        \lstdefinestyle{MatlabStyle_inc}{   % 插入代码的样式
            language=Matlab,
            basicstyle=\small\ttfamily\codefont,    % ttfamily 确保等宽 
            breakatwhitespace=false,
            breaklines=true,
            captionpos=b,
            keepspaces=true,
            numbers=left,
            numbersep=15pt,
            showspaces=false,
            showstringspaces=false,
            showtabs=false,
            tabsize=2,
            xleftmargin=15pt,   % 左边距
            %frame=single, % single 为包围式单线框
            frame=shadowbox,    % shadowbox 为带阴影包围式单线框效果
            %escapeinside=``,   % 允许在代码块中使用 LaTeX 命令 (此行无用)
            %frameround=tttt,    % tttt 表示四个角都是圆角
            framextopmargin=0pt,    % 边框上边距
            framexbottommargin=0pt, % 边框下边距
            framexleftmargin=5pt,   % 边框左边距
            framexrightmargin=5pt,  % 边框右边距
            rulesepcolor=\color{red!20!green!20!blue!20}, % 阴影框颜色设置
            %backgroundcolor=\color{blue!10}, % 背景颜色
        }
        \lstdefinestyle{MatlabStyle_src}{   % 插入代码的样式
            language=Matlab,
            basicstyle=\small\ttfamily\codefont,    % ttfamily 确保等宽 
            breakatwhitespace=false,
            breaklines=true,
            captionpos=b,
            keepspaces=true,
            numbers=left,
            numbersep=15pt,
            showspaces=false,
            showstringspaces=false,
            showtabs=false,
            tabsize=2,
        }
        \newtcblisting{matlablisting}{
            %arc=2pt,        % 圆角半径
            % 调整代码在 listing 中的位置以和引入文件时的格式相同
            top=0pt,
            bottom=0pt,
            left=-5pt,
            right=-5pt,
            listing only,   % 此句不能删去
            listing style=MatlabStyle_src,
            breakable,
            colback=white,   % 选一个合适的颜色
            colframe=black!0,   % 感叹号后跟不透明度 (为 0 时完全透明)
        }
        \lstset{
            style=MatlabStyle_inc,
        }



% table 支持
    \usepackage{booktabs}   % 宏包:三线表
    %\usepackage{tabularray} % 宏包:表格排版
    %\usepackage{longtable}  % 宏包:长表格
    %\usepackage[longtable]{multirow} % 宏包:multi 行列


% figure 设置
\usepackage{graphicx}   % 支持 jpg, png, eps, pdf 图片 
\usepackage{float}      % 支持 H 选项
\usepackage{svg}        % 支持 svg 图片
\usepackage{subcaption} % 支持子图
\svgsetup{
        % 指向 inkscape.exe 的路径
       inkscapeexe = C:/aa_MySame/inkscape/bin/inkscape.exe, 
        % 一定程度上修复导入后图片文字溢出几何图形的问题
       inkscapelatex = false                 
   }

% 图表进阶设置
    \usepackage{caption}    % 图注、表注
        \captionsetup[figure]{name=图}  
        \captionsetup[table]{name=表}
        \captionsetup{
            labelfont=bf, % 设置标签为粗体
            textfont=bf,  % 设置文本为粗体
            font=small  
        }
    \usepackage{float}     % 图表位置浮动设置 
        % \floatstyle{plaintop} % 设置表格标题在表格上方
        % \restylefloat{table}  % 应用设置


% 圆圈序号自定义
    \newcommand*\circled[1]{\tikz[baseline=(char.base)]{\node[shape=circle,draw,inner sep=0.8pt, line width = 0.03em] (char) {\small \bfseries #1};}}   % TikZ solution


% 列表设置
    \usepackage{enumitem}   % 宏包:列表环境设置
        \setlist[enumerate]{
            label=\bfseries(\arabic*) ,   % 设置序号样式为加粗的 (1) (2) (3)
            ref=\arabic*, % 如果需要引用列表项,这将决定引用格式(这里仍然使用数字)
            itemsep=0pt, parsep=0pt, topsep=0pt, partopsep=0pt, leftmargin=3.5em} 
        \setlist[itemize]{itemsep=0pt, parsep=0pt, topsep=0pt, partopsep=0pt, leftmargin=3.5em}
        \newlist{circledenum}{enumerate}{1} % 创建一个新的枚举环境  
        \setlist[circledenum,1]{  
            label=\protect\circled{\arabic*}, % 使用 \arabic* 来获取当前枚举计数器的值,并用 \circled 包装它  
            ref=\arabic*, % 如果需要引用列表项,这将决定引用格式(这里仍然使用数字)
            itemsep=0pt, parsep=0pt, topsep=0pt, partopsep=0pt, leftmargin=3.5em
        }  

% 文章默认字体设置
    \usepackage{fontspec}   % 宏包:字体设置
        \setmainfont{STKaiti}    % 设置中文字体为宋体字体
        \setCJKmainfont[AutoFakeBold=3]{STKaiti} % 设置加粗字体为 STKaiti 族,AutoFakeBold 可以调整字体粗细
        \setmainfont{Times New Roman} % 设置英文字体为Times New Roman


% 其它设置
    % 脚注设置
    \renewcommand\thefootnote{\ding{\numexpr171+\value{footnote}}}
    % 参考文献引用设置
        \bibliographystyle{unsrt}   % 设置参考文献引用格式为unsrt
        \newcommand{\upcite}[1]{\textsuperscript{\cite{#1}}}     % 自定义上角标式引用
    % 文章序言设置
        \newcommand{\cnabstractname}{序言}
        \newenvironment{cnabstract}{%
            \par\Large
            \noindent\mbox{}\hfill{\bfseries \cnabstractname}\hfill\mbox{}\par
            \vskip 2.5ex
            }{\par\vskip 2.5ex}


% 各级标题自定义设置
    \usepackage{titlesec}   
    % chapter
        \titleformat{\chapter}[hang]{\normalfont\Large\bfseries\centering}{Homework \thechapter }{10pt}{}
        \titlespacing*{\chapter}{0pt}{-30pt}{10pt} % 控制上方空白的大小
    % section
        \titleformat{\section}[hang]{\normalfont\large\bfseries}{\thesection}{8pt}{}
    % subsection
        %\titleformat{\subsubsection}[hang]{\normalfont\bfseries}{}{8pt}{}
    % subsubsection
        %\titleformat{\subsubsection}[hang]{\normalfont\bfseries}{}{8pt}{}


% >> ------------------ 文章宏包及相关设置 ------------------ << %
% ------------------------------------------------------------- %



% ----------------------------------------------------------- %
% >> --------------------- 文章信息区 --------------------- << %
% 页眉页脚设置

\usepackage{fancyhdr}   %宏包:页眉页脚设置
    \pagestyle{fancy}
    \fancyhf{}
    \cfoot{\thepage}
    \renewcommand\headrulewidth{1pt}
    \renewcommand\footrulewidth{0pt}
    \chead{模式识别与机器学习作业,\ 尹超,\ 2023K8009926003}
    \lhead{Homework}
    \rhead{yinchao23@mails.ucas.ac.cn}

%文档信息设置
\title{模式识别与机器学习作业\\ PRML Homework}
\author{尹超\\ \footnotesize 中国科学院大学,北京 10004 \\ \footnotesize University of Chinese Academy of Sciences, Beijing 100049, China}
\date{\footnotesize 2024.8 -- 2025.1}
% >> --------------------- 文章信息区 --------------------- << %
% ----------------------------------------------------------- %     


% 开始编辑文章

\begin{document}
\zihao{5}           % 设置全文字号大小

% --------------------------------------------------------------- %
% >> --------------------- 封面序言与目录 --------------------- << %
% 封面
    \maketitle\newpage  
    \pagenumbering{Roman} % 页码为大写罗马数字
    \thispagestyle{fancy}   % 显示页码、页眉等

% 序言
    \begin{cnabstract}\normalsize 
        本文为笔者模式识别与机器学习的作业。\par
        望老师批评指正。
    \end{cnabstract}
    \addcontentsline{toc}{chapter}{序言} % 手动添加为目录

% % 不换页目录
%     \setcounter{tocdepth}{0}
%     \noindent\rule{\textwidth}{0.1em}   % 分割线
%     \noindent\begin{minipage}{\textwidth}\centering 
%         \vspace{1cm}
%         \tableofcontents\thispagestyle{fancy}   % 显示页码、页眉等   
%     \end{minipage}  
%     \addcontentsline{toc}{chapter}{目录} % 手动添加为目录

% 目录
\setcounter{tocdepth}{4}                % 目录深度(为1时显示到section)
\tableofcontents                        % 目录页
\addcontentsline{toc}{chapter}{目录}    % 手动添加此页为目录
\thispagestyle{fancy}                   % 显示页码、页眉等 

% 收尾工作
    \newpage    
    \pagenumbering{arabic} 

% >> --------------------- 封面序言与目录 --------------------- << %
% --------------------------------------------------------------- %


\chapter{分类与估计问题的解答}


\section{问题1:贝叶斯分类与风险决策}

\subsection{问题描述}
对于 \( c \) 类分类问题,假设各类先验概率为 \( P(\omega_i), i=1,\ldots,c \),类条件概率密度为 \( p(\mathbf{x} \mid \omega_i), i=1,\ldots,c \)(\(\mathbf{x}\) 为样本特征向量),将第 \( j \) 类样本误判为第 \( i \) 类的风险为 \( \lambda_{ij} \)。  
\begin{enumerate}
    \item[(1)] 写出贝叶斯最小错误率决策和最小风险决策的决策规则。
    \item[(2)] 引入拒识选项(第 \( c+1 \) 类)。假设正确分类风险 \( \lambda(\alpha_i \mid \omega_j) = 0 (i=j) \);错误分类风险 \( \lambda(\alpha_i \mid \omega_j) = \lambda_s (i \neq j) \);拒识风险 \( \lambda(\alpha_i \mid \omega_j) = \lambda_r (i = c+1) \)。写出带拒识的最小风险决策规则。
    \item[(3)] 说明拒识的意义。
\end{enumerate}

\subsection{解答}
\begin{itemize}
    \item[(1)] \textbf{决策规则}
    \begin{itemize}
        \item \textbf{贝叶斯最小错误率决策}:  
        对样本 \(\mathbf{x}\),选择最大化后验概率的类别:  
        \[
        \hat{\omega} = \arg\max_{\omega_i} P(\omega_i \mid \mathbf{x}) = \arg\max_{\omega_i} p(\mathbf{x} \mid \omega_i) P(\omega_i)
        \]
        \item \textbf{最小风险决策}:  
        选择最小化期望风险的类别:  
        \[
        \hat{\omega} = \arg\min_{\omega_i} \sum_{j=1}^c \lambda_{ij} P(\omega_j \mid \mathbf{x})
        \]
    \end{itemize}
    \item[(2)] \textbf{带拒识的最小风险决策}  
    拒识为 \(\alpha_{c+1}\)。分类风险:  
    \[
    R(\alpha_i \mid \mathbf{x}) = \lambda_s (1 - P(\omega_i \mid \mathbf{x})), \quad i=1,\ldots,c
    \]  
    拒识风险:  
    \[
    R(\alpha_{c+1} \mid \mathbf{x}) = \lambda_r
    \]  
    决策规则:  
    \[
    \hat{\alpha} = \arg\min_{\alpha_k} R(\alpha_k \mid \mathbf{x}), \quad k = 1, \ldots, c+1
    \]  
    若 \( P(\omega_i \mid \mathbf{x}) > 1 - \frac{\lambda_r}{\lambda_s} \),分类为 \(\omega_i\);否则拒识。
    \item[(3)] \textbf{拒识的意义}  
    拒识允许分类器在不确定性高时避免决策,降低误分类风险。例如,在医疗诊断中,拒识可避免错误治疗。
\end{itemize}

\section{问题2:高斯分布下的分类}

\subsection{问题描述}
对于 \( c \) 类分类问题,特征向量 \(\mathbf{x} \in \mathbb{R}^d \),假设各类先验概率相等,类条件概率密度为 \( d \) 维高斯分布。  
\begin{enumerate}
    \item[(1)] 写出类条件概率密度函数的数学形式。
    \item[(2)] 写出协方差矩阵相等和不等情况下的最小错误率判别函数。
    \item[(3)] 两类等协方差情况下,决策面特点?何时通过两类均值中点?
    \item[(4)] 当协方差矩阵奇异时,如何克服?
\end{enumerate}

\subsection{解答}
\begin{itemize}
    \item[(1)] \textbf{类条件概率密度}  
    \[
    p(\mathbf{x} \mid \omega_i) = \frac{1}{(2\pi)^{d/2} |\Sigma_i|^{1/2}} \exp\left( -\frac{1}{2} (\mathbf{x} - \mu_i)^T \Sigma_i^{-1} (\mathbf{x} - \mu_i) \right)
    \]
    \item[(2)] \textbf{判别函数}  
    \begin{itemize}
        \item \textbf{协方差相等} (\(\Sigma_i = \Sigma\)):  
        \[
        g_i(\mathbf{x}) = \mathbf{w}^T \mathbf{x} + w_{i0}, \quad \mathbf{w} = \Sigma^{-1} \mu_i, \quad w_{i0} = -\frac{1}{2} \mu_i^T \Sigma^{-1} \mu_i
        \]
        \item \textbf{协方差不等}:  
        \[
        g_i(\mathbf{x}) = -\frac{1}{2} \log |\Sigma_i| - \frac{1}{2} (\mathbf{x} - \mu_i)^T \Sigma_i^{-1} (\mathbf{x} - \mu_i)
        \]
    \end{itemize}
    \item[(3)] \textbf{决策面特点}  
    为超平面;当先验概率相等时,通过均值中点 \(\frac{\mu_1 + \mu_2}{2}\)。
    \item[(4)] \textbf{克服奇异协方差}  
    在协方差矩阵对角线上加小正数(岭正则化)。
\end{itemize}

\section{问题3:隐马尔可夫模型示例}

\subsection{问题描述}
举出生活中符合隐马尔可夫模型(HMM)的例子,说明隐状态和观测变量含义,及观测序列评价、状态序列解码的实际问题。

\subsection{解答}
\textbf{示例}:语音识别  
\begin{itemize}
    \item \textbf{隐状态}:音素序列(如 /k/, /æ/, /t/ 表示“cat”)。
    \item \textbf{观测变量}:音频信号的 MFCC 特征。
    \item \textbf{观测序列评价}:计算音频对应特定单词的概率。
    \item \textbf{状态序列解码}:确定音频的最可能音素序列,识别单词。
\end{itemize}

\section{问题4:参数与非参数估计}

\subsection{问题描述}
\begin{enumerate}
    \item[(1)] 总结最大似然估计、贝叶斯估计、k近邻估计和Parzen估计的原理和步骤。
    \item[(2)] 一维特征空间样本点 \{-1, 2, 3, 6\},服从均匀分布,用最大似然法估计概率密度函数。
    \item[(3)] 样本服从高斯分布,协方差 \(\Sigma\) 已知,均值 \(\mu\) 未知,先验为 \(\mathcal{N}(m_0, \Sigma_0)\),写出贝叶斯估计过程。
    \item[(4)] 说明k近邻密度估计与分类规则的关系。
\end{enumerate}

\subsection{解答}
\begin{itemize}
    \item[(1)] \textbf{估计方法}  
    \begin{itemize}
        \item \textbf{最大似然估计(MLE)}:最大化似然函数 \( L(\theta) = p(\mathbf{x} \mid \theta) \)。步骤:定义似然,对数化,求导,解参数。
        \item \textbf{贝叶斯估计}:使用后验 \( p(\theta \mid \mathbf{x}) \propto p(\mathbf{x} \mid \theta) p(\theta) \),取均值或众数。步骤:指定先验,计算后验,估计参数。
        \item \textbf{k近邻估计}:基于样本局部密度估计。步骤:选择 \( k \),计算距离,估计密度为 \(\frac{k}{nV}\)。
        \item \textbf{Parzen窗估计}:使用核函数估计密度。步骤:选择核和带宽,累加样本贡献。
    \end{itemize}
    \item[(2)] \textbf{均匀分布的 MLE}  
    样本:\{-1, 2, 3, 6\},假设 \( x \sim U(a, b) \)。  
    似然:\( L(a, b) = \frac{1}{(b - a)^4} \),约束 \( a \leq -1 \),\( b \geq 6 \)。  
    最大化得:\(\hat{a} = -1\), \(\hat{b} = 6\)。  
    密度:\( p(x) = \frac{1}{7} \),\( -1 \leq x \leq 6 \)。
    \item[(3)] \textbf{贝叶斯估计}  
    后验:  
    \[
    p(\mu \mid \mathbf{x}) = \mathcal{N}\left( \left( \Sigma_0^{-1} + n \Sigma^{-1} \right)^{-1} \left( \Sigma_0^{-1} m_0 + n \Sigma^{-1} \bar{\mathbf{x}} \right), \left( \Sigma_0^{-1} + n \Sigma^{-1} \right)^{-1} \right)
    \]  
    估计:\(\hat{\mu} = \left( \Sigma_0^{-1} + n \Sigma^{-1} \right)^{-1} \left( \Sigma_0^{-1} m_0 + n \Sigma^{-1} \bar{\mathbf{x}} \right)\)。
    \item[(4)] \textbf{k近邻关系}  
    密度估计通过局部样本统计密度,分类通过多数投票;两者均依赖局部信息,密度估计可间接支持分类。
\end{itemize}

\section{问题5:USPS数据集分类}

\subsection{问题描述}
在 USPS 数据集中,选择两个类别(例如数字 0 和 1),实现贝叶斯分类器:  
\begin{enumerate}
    \item[(1)] 假设两类协方差矩阵相等,使用线性判别函数分类,并与 Fisher 判别准则的线性分类器比较。
    \item[(2)] 分别估计两类协方差矩阵,使用二次判别函数分类。
    \item[(3)] 分析两种情况的结果。
\end{enumerate}

\textbf{附录:USPS 数据集}  
手写数字数据集,包含 7291 个训练样本和 2007 个测试样本,每样本为 16×16 灰度图像(256 维)。数据以 HDF5 格式提供,包含 \texttt{train} 和 \texttt{test} 组,各有 \texttt{data}(特征)和 \texttt{target}(标签)。

读取代码:
\begin{lstlisting}
import h5py
with h5py.File(path, 'r') as hf:
    train = hf.get('train')
    X_tr = train.get('data')[:]
    y_tr = train.get('target')[:]
    test = hf.get('test')
    X_te = test.get('data')[:]
    y_te = test.get('target')[:]
\end{lstlisting}

\subsection{解答}

\subsubsection{方法}
\begin{itemize}
    \item \textbf{数据预处理}:
    \begin{itemize}
        \item 使用 \texttt{h5py} 加载数据集。
        \item 选择数字 0 和 1,过滤训练和测试数据。
        \item 特征维度:256 维向量。
    \end{itemize}
    \item \textbf{第一部分:线性判别分析(LDA)与 Fisher 判别}:
    \begin{itemize}
        \item \textbf{假设}:两类共享协方差矩阵。
        \item \textbf{LDA}:线性判别函数:
        \[
        g(\mathbf{x}) = \mathbf{w}^T \mathbf{x} + w_0
        \]
        其中:
        \[
        \mathbf{w} = \Sigma^{-1} (\mu_1 - \mu_0), \quad w_0 = -\frac{1}{2} (\mu_0 + \mu_1)^T \Sigma^{-1} (\mu_1 - \mu_0)
        \]
        \item \textbf{Fisher 判别}:投影向量:
        \[
        \mathbf{w} = S_W^{-1} (\mu_1 - \mu_0)
        \]
        在此情况下与 LDA 等价。
        \item \textbf{注意}:特征维度(256)可能导致协方差矩阵奇异,需正则化。
    \end{itemize}
    \item \textbf{第二部分:二次判别分析(QDA)}:
    \begin{itemize}
        \item \textbf{假设}:两类协方差矩阵不同。
        \item \textbf{判别函数}:
        \[
        g_i(\mathbf{x}) = -\frac{1}{2} \log |\Sigma_i| - \frac{1}{2} (\mathbf{x} - \mu_i)^T \Sigma_i^{-1} (\mathbf{x} - \mu_i)
        \]
        \item \textbf{实现}:分别计算协方差矩阵,加入正则化。
    \end{itemize}
    \item \textbf{评估与分析}:
    \begin{itemize}
        \item 在测试集上计算准确率。
        \item 讨论高维和小样本的影响。
    \end{itemize}
\end{itemize}

\subsubsection{Python 代码实现}
\begin{lstlisting}
import h5py
import numpy as np
from numpy.linalg import pinv, det

# 加载 USPS 数据集
with h5py.File('usps.h5', 'r') as hf:
    train = hf.get('train')
    X_tr = train.get('data')[:]  # 训练数据
    y_tr = train.get('target')[:]  # 训练标签
    test = hf.get('test')
    X_te = test.get('data')[:]  # 测试数据
    y_te = test.get('target')[:]  # 测试标签

# 过滤类别 0 和 1
mask_tr = (y_tr == 0) | (y_tr == 1)
X_tr = X_tr[mask_tr]
y_tr = y_tr[mask_tr]
mask_te = (y_te == 0) | (y_te == 1)
X_te = X_te[mask_te]
y_te = y_te[mask_te]

# 第一部分:LDA(协方差矩阵相等)
# 计算每类均值
mu0 = np.mean(X_tr[y_tr == 0], axis=0)  # 类别 0 均值
mu1 = np.mean(X_tr[y_tr == 1], axis=0)  # 类别 1 均值

# 计算共享协方差矩阵并正则化
epsilon = 1e-5
d = X_tr.shape[1]
Sigma = np.cov(X_tr.T) + epsilon * np.eye(d)

# 计算线性判别参数
w = pinv(Sigma) @ (mu1 - mu0)
w0 = -0.5 * (mu0 + mu1).T @ pinv(Sigma) @ (mu1 - mu0)

# 测试集预测
scores = X_te @ w + w0
y_pred_lda = (scores > 0).astype(int)
accuracy_lda = np.mean(y_pred_lda == y_te)
print(f'LDA 准确率: {accuracy_lda:.4f}')

# Fisher 判别:与 LDA 等价,省略

# 第二部分:QDA(分别估计协方差)
# 计算每类协方差矩阵并正则化
Sigma0 = np.cov(X_tr[y_tr == 0].T) + epsilon * np.eye(d)
Sigma1 = np.cov(X_tr[y_tr == 1].T) + epsilon * np.eye(d)

# 计算逆矩阵和行列式
inv_Sigma0 = pinv(Sigma0)
inv_Sigma1 = pinv(Sigma1)
log_det_Sigma0 = np.log(det(Sigma0))
log_det_Sigma1 = np.log(det(Sigma1))

# 定义二次判别函数
def g(x, mu, inv_Sigma, log_det):
    diff = x - mu
    return -0.5 * log_det - 0.5 * diff.T @ inv_Sigma @ diff

# 测试集预测
y_pred_qda = []
for x in X_te:
    g0 = ['$x = np.linspace(0, 10, 100)
y = np.sin(x)

plt.figure(figsize=(8, 6))
plt.plot(x, y, 'b-', label='Sine wave')
plt.title('Simple Sine Wave')
plt.xlabel('X axis')
plt.ylabel('Y axis')
plt.grid(True)
plt.legend()

plt.savefig('sine_wave.png') g(x, mu0, inv_Sigma0, log_det_Sigma0)
    g1 = g(x, mu1, inv_Sigma1, log_det_Sigma1)
    y_pred_qda.append(1 if g1 > g0 else 0)

accuracy_qda = np.mean(np.array(y_pred_qda) == y_te)
print(f'QDA 准确率: {accuracy_qda:.4f}')
\end{lstlisting}

\subsubsection{代码说明}
\begin{itemize}
    \item \textbf{数据加载与过滤}:使用 \texttt{h5py} 读取数据,过滤类别 0 和 1。
    \item \textbf{LDA 实现}:计算均值和共享协方差,加入正则化,执行线性分类。
    \item \textbf{QDA 实现}:计算每类协方差,计算二次判别函数。
    \item \textbf{Fisher 判别}:与 LDA 等价,未单独实现。
\end{itemize}

\subsubsection{结果分析与讨论}
\begin{itemize}
    \item \textbf{数据特性}:
    \begin{itemize}
        \item 训练集约 1400-1500 个样本,每类约 700 个。
        \item 特征维度 256,图像特征相关性高,需正则化。
    \end{itemize}
    \item \textbf{LDA 与 Fisher 判别}:
    \begin{itemize}
        \item LDA 使用线性边界,适合 0 和 1 的明显差异。
        \item Fisher 判别在此等价。
    \end{itemize}
    \item \textbf{QDA 性能}:
    \begin{itemize}
        \item 优势:捕捉协方差差异。
        \item 挑战:高维小样本导致过拟合。
    \end{itemize}
    \item \textbf{预期结果}:
    \begin{itemize}
        \item 准确率可能 >90\%。
        \item LDA 因参数少更稳定,QDA 可能因过拟合略逊。
    \end{itemize}
    \item \textbf{改进建议}:
    \begin{itemize}
        \item 使用 PCA 降维。
        \item 优化正则化参数 \(\epsilon\)。
        \item 考虑数据增强。
    \end{itemize}
\end{itemize}

\subsubsection{总结}
LDA 和 QDA 分类器成功分类 USPS 数据集的数字 0 和 1。LDA 因简单性更鲁棒,QDA 因高维限制可能过拟合。两者对 0 和 1 均表现良好,具体准确率需运行验证。

% 附录 A
\chapter*{附录 A. 中英文对照表}\addcontentsline{toc}{chapter}{附录 A. 中英文对照表}   
\thispagestyle{plain} 
\setcounter{section}{0}   
\renewcommand\thesection{A.\arabic{section}}   
\renewcommand{\thefigure}{A.\arabic{figure}} 
\renewcommand{\thetable}{A.\arabic{table}}

\section{中英文对照表}
\begin{multicols}{2}  

\begin{table}[H]
\centering
\caption{\textbf{中英文对照表}}
\begin{tabular}{ll}
\toprule
英文 & 中文 \\
\midrule
Bayes classification & 贝叶斯分类 \\
decision rule & 决策规则 \\
minimum error rate & 最小错误率 \\
minimum risk & 最小风险 \\
rejection option & 拒识选项 \\
Gaussian distribution & 高斯分布 \\
covariance matrix & 协方差矩阵 \\
discriminant function & 判别函数 \\
decision boundary & 决策边界 \\
Hidden Markov Model & 隐马尔可夫模型 \\
hidden state & 隐状态 \\
observation sequence & 观测序列 \\
maximum likelihood estimation & 最大似然估计 \\
Bayesian estimation & 贝叶斯估计 \\
k-Nearest Neighbor & k近邻 \\
Parzen window & Parzen窗 \\
linear discriminant & 线性判别 \\
quadratic discriminant & 二次判别 \\
prior probability & 先验概率 \\
posterior probability & 后验概率 \\
\bottomrule
\end{tabular}
\end{table}

\end{multicols}


\end{document}